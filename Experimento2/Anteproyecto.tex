\documentclass[letterpaper]{article}
\usepackage[utf8]{inputenc}
\usepackage[spanish]{babel}

\begin{document}

\title{Anteproyecto1\\Laboratorio de Microcontroladores}
\author{
 Marco Antonio Montero Chavarría Carné: A94000\\
  \and
  Francisco Mamaglande Molina Carné:B0-CLACOME\\  
}
\maketitle

\section{Investigación Previa}
Un semáforo inteligente es un sistema que tiene 2 partes importantes a poner atención, se tiene
un semáforo para el paso de carros que normalmente tiene 3 luces disponibles rojo para evitar el paso,
amarillo para indicar a los conductores que disminuyan la velocidad ya que el semáforo esta apunto de cambiar a rojo
y verde para indicar paso a los conductores. Además se tiene un segundo semáforo que esta destinado para los peatones
este posee dos luces, una roja que indica a los peatones que no deben cruzar la calle y el verde que indica que pueden
transitar.

Ahora bien cuál es la magia de una semáforo inteligente?. La idea es que se tiene un botón de paso en el semáforo
peatonal y de ser presionado un tiempo después el semáforo de automoviles debe cambiar de estado a un estado de no paso
y el semáforo peatonal deberá cambiar a un estado de paso para los peatones.

Este sistema se utiliza normalmente en la mayoría de cruces peatonales en el sistema vial. Naturalmente se pretende
que si el botón de paso no es presionado el semáforo de automóviles no cambia de estado ya que sería inecesario.

Para este trabajo se tiene un semáforo de carros un poco más simple con solo 2 luces y una luz peatonal con también 2 luces.
Se requiere usar un stm32f4 y recrear la mecánica de un semáforo inteligente utilizando las luces disponibles en el microcontrolador.


\section{Solución Propuesta}
Utilizando código de ejemplos del stm32f4, se crea por pedazos el código para el semáforo inteligente. Se requiere manejar 
los leds, crear un temporizador, y activar un botón.


\section{Procedimiento}

\section{Observaciones y recomendaciones}

\bibliographystyle{alpha} 
\bibliography{refs}



\end{document}
