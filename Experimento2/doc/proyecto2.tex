\documentclass[letterpaper]{article}
\usepackage[T1]{fontenc}
%\usepackage[utf8]{inputenc}
\usepackage[ansinew]{inputenc}
\usepackage[spanish]{babel}
\usepackage{graphicx}
\usepackage{listings}
\title{Proyecto 2\\ Laboratorio de Microcontroladores Temporizador y GPIOs}
\author{
Marco Antonio Montero Chavarr��a Carn�: A94000\\
 \and
Francisco Molina Carn�: B14194\\
}

\begin{document}
\maketitle
\section{Problema}
 Desarrollar un "sem�foro inteligente" simplificado utlizando los leds, la botonera, y los temporizadores del microcontrolador STM32F4,e los cuatro leds disponibles en el microcontrolador los leds LD3 y LD5 representar�n el sem�foro vehicular, mientras que los LD4 y LD6 ser�an el sem�foro peatonal. El bot�n B1 (user\_button) ser�a el puerto con el que el usuario solicitar� la activaci�n de las luces peatonales.
La asignaci�n de funcionalidades de los leds es la siguiente:

\begin{itemize}
\item LD3:paso de veh�culos.
\item LD5:veh�culos detenidos.
\item LD4:paso de peatones.
\item LD6:peatones deternidos
\end{itemize}

 El bot�n se puede presionar inclusive antes que se acaben los 10s  del funcionamiento de LD3, pero deber�a permanecer encendido hasta que terminen los 10s. Si no se presiona el bot�n LD3 deber�a permancer
encendido indefinidamente.
La temporizaci�n de los leds se debe realizar utilizando uno de los temporizadores del microcontrolador.
Se recomienda utilizar el Advanced Timer 1 con interrupciones de tiempo habilitados.
Como innovaci�n se podr�an agregar leds externos para representar adecuadamente con colores verde y rojo ambos sem�foros, o agregar una luz amarilla externa con su respectivo control adicional. En cualquiera de los se deber�an utilizar resistencias de pull-up para los leds alimentados por el mismo micro-controlador para garantizar la seguridad de la tarjeta.

\section{Soluci�n Propuesta}
 
Generar un c�digo para el sem�foro utilizando el c�digo visto en clase que contiene m�todos de definici�n para el bot�n, para el temporizador y para encender los leds.

\section{Procedimiento}
Se utiliz� el archivo glove.c en la carpeta scr\_glove de opencoroco. Se removi� las partes del c�digo inecesarias para el sem�foro y se a�adi� lo necesario para darle funcionalidad al temporizador, al bot�n y a los leds.

Para el temporizador se utiliz� el siguiente c�digo:

\begin{lstlisting}

void DTC_SVM_tim_init(void)
{
/* Enable TIM1 clock. and Port E clock (for outputs) */
rcc_peripheral_enable_clock(&RCC_APB2ENR, RCC_APB2ENR_TIM1EN);
rcc_peripheral_enable_clock(&RCC_AHB1ENR, RCC_AHB1ENR_IOPEEN);

//Set TIM1 channel (and complementary) 
//output to alternate function push-pull'.
//f4 TIM1=> GIO9: CH1, GPIO11: CH2, GPIO13: CH3
//f4 TIM1=> GIO8: CH1N, GPIO10: CH2N, GPIO12: CH3N
gpio_mode_setup(GPIOE, GPIO_MODE_AF,GPIO_PUPD_NONE,GPIO9 | GPIO11 | GPIO13);
gpio_set_af(GPIOE, GPIO_AF1, GPIO9 | GPIO11 | GPIO13);
gpio_mode_setup(GPIOE, GPIO_MODE_AF,GPIO_PUPD_NONE,GPIO8 | GPIO10 | GPIO12);
gpio_set_af(GPIOE, GPIO_AF1, GPIO8 | GPIO10 | GPIO12);

	/* Enable TIM1 commutation interrupt. */
	//nvic_enable_irq(NVIC_TIM1_TRG_COM_TIM11_IRQ);	//f4

	/* Reset TIM1 peripheral. */
	timer_reset(TIM1);

	/* Timer global mode:
	 * - No divider
	 * - Alignment edge
	 * - Direction up
	 */
	timer_set_mode(TIM1, TIM_CR1_CKD_CK_INT, 
//For dead time and filter sampling, not important for now.
TIM_CR1_CMS_CENTER_1,	
//TIM_CR1_CMS_EDGE
						
//TIM_CR1_CMS_CENTER_1
						
//TIM_CR1_CMS_CENTER_2
						
//TIM_CR1_CMS_CENTER_3 la frequencia del pwm se divide a la mitad. 
//(frecuencia senoidal)
	TIM_CR1_DIR_UP);

	timer_set_prescaler(TIM1, PRESCALE); 
//1 = disabled (max speed)
	timer_set_repetition_counter(TIM1, 0); 
//disabled
	timer_enable_preload(TIM1);
	timer_continuous_mode(TIM1);

	/* Period (32kHz). */
	timer_set_period(TIM1, PWM_PERIOD_ARR); 
//ARR (value compared against main counter to reload counter 
//aka: period of counter)

	/* Configure break and deadtime. */
//timer_set_deadtime(TIM1, deadtime_percentage*pwm_period_ARR);
//timer_set_deadtime(TIM1, 1100.0f*PWM_PERIOD_ARR);
//timer_set_deadtime(TIM1, DEAD_TIME_PERCENTAGE*PWM_PERIOD_ARR);
	timer_set_enabled_off_state_in_idle_mode(TIM1);
	timer_set_enabled_off_state_in_run_mode(TIM1);
	timer_disable_break(TIM1);
	timer_set_break_polarity_high(TIM1);
	timer_disable_break_automatic_output(TIM1);
	timer_set_break_lock(TIM1, TIM_BDTR_LOCK_OFF);

	/* Disable outputs. */
	timer_disable_oc_output(TIM1, TIM_OC1);
	timer_disable_oc_output(TIM1, TIM_OC1N);
	timer_disable_oc_output(TIM1, TIM_OC2);
	timer_disable_oc_output(TIM1, TIM_OC2N);
	timer_disable_oc_output(TIM1, TIM_OC3);
	timer_disable_oc_output(TIM1, TIM_OC3N);

	/* -- OC1 and OC1N configuration -- */
	/* Configure global mode of line 1. */
	timer_enable_oc_preload(TIM1, TIM_OC1);
	timer_set_oc_mode(TIM1, TIM_OC1, TIM_OCM_PWM1);
	/* Configure OC1. */
	timer_set_oc_polarity_high(TIM1, TIM_OC1);
	timer_set_oc_idle_state_unset(TIM1, TIM_OC1); 
//When idle (braked) put 0 on output
	/* Configure OC1N. */
	timer_set_oc_polarity_high(TIM1, TIM_OC1N);
	timer_set_oc_idle_state_unset(TIM1, TIM_OC1N);
	/* Set the capture compare value for OC1. */
timer_set_oc_value(TIM1, TIM_OC1, INIT_DUTY*PWM_PERIOD_ARR);
//initial_duty_cycle*pwm_period_ARR);

	/* -- OC2 and OC2N configuration -- */
	/* Configure global mode of line 2. */
	timer_enable_oc_preload(TIM1, TIM_OC2);
	timer_set_oc_mode(TIM1, TIM_OC2, TIM_OCM_PWM1);
	/* Configure OC2. */
	timer_set_oc_polarity_high(TIM1, TIM_OC2);
	timer_set_oc_idle_state_unset(TIM1, TIM_OC2);
	/* Configure OC2N. */
	timer_set_oc_polarity_high(TIM1, TIM_OC2N);
	timer_set_oc_idle_state_unset(TIM1, TIM_OC2N);
	/* Set the capture compare value for OC2. */
timer_set_oc_value(TIM1, TIM_OC2, INIT_DUTY*PWM_PERIOD_ARR);
//initial_duty_cycle*pwm_period_ARR);

	/* -- OC3 and OC3N configuration -- */
	/* Configure global mode of line 3. */
	timer_enable_oc_preload(TIM1, TIM_OC3);
	timer_set_oc_mode(TIM1, TIM_OC3, TIM_OCM_PWM1);
	/* Configure OC3. */
	timer_set_oc_polarity_high(TIM1, TIM_OC3);
	timer_set_oc_idle_state_unset(TIM1, TIM_OC3);
	/* Configure OC3N. */
	timer_set_oc_polarity_high(TIM1, TIM_OC3N);
	timer_set_oc_idle_state_unset(TIM1, TIM_OC3N);
	/* Set the capture compare value for OC3. */
timer_set_oc_value(TIM1, TIM_OC3, INIT_DUTY*PWM_PERIOD_ARR);
//initial_duty_cycle*pwm_period_ARR);//100);

	/* Reenable outputs. */
	timer_enable_oc_output(TIM1, TIM_OC1);
	timer_enable_oc_output(TIM1, TIM_OC1N);
	timer_enable_oc_output(TIM1, TIM_OC2);
	timer_enable_oc_output(TIM1, TIM_OC2N);
	timer_enable_oc_output(TIM1, TIM_OC3);
	timer_enable_oc_output(TIM1, TIM_OC3N);

	/* ---- */

	/* ARR reload enable. */
	timer_enable_preload(TIM1);

/*
 * Enable preload of complementary channel configurations and
 * update on COM event.
 */
	//timer_enable_preload_complementry_enable_bits(TIM1);
	timer_disable_preload_complementry_enable_bits(TIM1);

	/* Enable outputs in the break subsystem. */
	timer_enable_break_main_output(TIM1);

/* Generate update event to reload all registers before starting*/
	timer_generate_event(TIM1, TIM_EGR_UG);

	/* Counter enable. */
	timer_enable_counter(TIM1);

	/* Enable commutation interrupt. */
	//timer_enable_irq(TIM1, TIM_DIER_COMIE);

	/*********/
	/*Capture compare interrupt*/

	//enable capture compare interrupt
	timer_enable_update_event(TIM1);

	/* Enable commutation interrupt. */
	//timer_enable_irq(TIM1, TIM_DIER_CC1IE);	
//Capture/compare 1 interrupt enable
	/* Enable commutation interrupt. */
	//timer_enable_irq(TIM1, TIM_DIER_CC1IE);
	timer_enable_irq(TIM1, TIM_DIER_UIE);
	nvic_enable_irq(NVIC_TIM1_UP_TIM10_IRQ);
}

void tim1_up_tim10_isr(void) 
{
  //Clear the update interrupt flag
  timer_clear_flag(TIM1,TIM_SR_UIF);
	
	
	
static int counter = 0;
counter +=1 ;
if(counter >=1000)
{
gpio_toggle(GPIOD,GPIO12);
counter=0;
}	
}


\end{lstlisting}

Para el bot�n se utiliz� el siguiente c�digo:

\begin{lstlisting}


void button_init(void)
{
 rcc_peripheral_enable_clock(&RCC_AHB1ENR, RCC_AHB1ENR_IOPAEN);
 gpio_mode_setup(GPIOA, GPIO_MODE_INPUT, GPIO_PUPD_NONE, GPIO0);
 gpio_set_af(GPIOA, GPIO_AF0,GPIO0);
}



Y dentro del m�todo void tim1_up_tim10_isr(void) \\

int button = 0;
	
	button = gpio_get(GPIOA,GPIO0);

	if(button > 0)
	{
	gpio_set(GPIOD,GPIO12);
	gpio_clear(GPIOD,GPIO13);
	gpio_set(GPIOD,GPIO14);
	gpio_clear(GPIOD,GPIO15);
	}
	else
	{
	gpio_clear(GPIOD,GPIO12);
	gpio_clear(GPIOD,GPIO13);
	gpio_clear(GPIOD,GPIO14);
	gpio_clear(GPIOD,GPIO15);

\end{lstlisting}

Para los leds se utiliz� el siguiente c�digo:

\begin{lstlisting}
void leds_init(void) 
{
  rcc_peripheral_enable_clock(&RCC_AHB1ENR, RCC_AHB1ENR_IOPDEN);
  gpio_mode_setup(GPIOD, GPIO_MODE_OUTPUT, GPIO_PUPD_NONE, GPIO12);
  gpio_mode_setup(GPIOD, GPIO_MODE_OUTPUT, GPIO_PUPD_NONE, GPIO13);
  gpio_mode_setup(GPIOD, GPIO_MODE_OUTPUT, GPIO_PUPD_NONE, GPIO14);
  gpio_mode_setup(GPIOD, GPIO_MODE_OUTPUT, GPIO_PUPD_NONE, GPIO15);
}

\end{lstlisting}
 Adem�s para inicializar los m�todos, estos deben ser llamados en el system init de la siguiente manera:

\begin{lstlisting}
void system_init(void) 
{
  rcc_clock_setup_hse_3v3(&hse_8mhz_3v3[CLOCK_3V3_168MHZ]);
  leds_init();
  button_init();
  DTC_SVM_tim_init();
  tim1_up_tim10_isr();

}
\end{lstlisting}
\section{Conclusiones}


\bibliographystyle{alpha} 
\bibliography{refs}



\end{document}
