\documentclass[letterpaper]{article}
\usepackage[utf8]{inputenc}
%\usepackage[ansinew]{inputenc}
\usepackage[spanish]{babel}
\usepackage{graphicx}

\begin{document}

\title{Propuesta proyecto final\\ Laboratorio de Microcontroladores}
\author{
 Marco Antonio Montero Chavarrí­a Carné: A94000\\
  \and
  Francisco Molina Carné: B14194\\  }
\maketitle

\section{Investigación Previa}
Se creará un controlador para una motocicleta eléctrica, capaz de cambiar la velocidad del motor y llevar un registro del manejo de consumo de energía así como de la velocidad instantanea para contar con datos de desempeño en diferentes situaciones. Adicionalmente se le agregarán funcionalidades como el sistema de luces y direccionales, entre otros que sean de interés en caso de cumplir los objetivos iniciales.\\

Una motocicleta o scooter eléctrica es una motocicleta que utiliza un motor eléctrico como medio de propulsión.
Al igual que el automóvil eléctrico no produce contaminación atmosférica ni contaminación sonora en el lugar de uso.
Se han desarrollado algunos prototipos de pilas de combustible, siendo algunos ejemplos el ENV de Intelligent Energy, 
el scooter de Honda que utiliza el Honda FC Stack, y el el Yamaha FC-AQEL.También se están en desarrollo prototipos de 
motocicletas híbridas con motor de gasolina y motor eléctrico. Algunos ejemplos son el Ecycle, y el Yamaha Gen-RYU.
No obstante, los modelos en producción son de baterías. \\

Entre los principales problemas que se puede tener con este trabajo es el tipo de alimentación que se debe usar para el motor de la motocicleta, así como el tipo de motor que se quiere utilizar. Para el tipo de motor, se requiere una alta eficiencia, por lo que se consideró como mejor opción un motor sin escobillas, que cuenta en general con mayor relación torque/watt y mayor durabilidad que un motor con escobillas. \\
Para el caso de las baterías, se necesita asimismo de alta eficiencia y mayor porcentaje de descarga, para poder manejar el consumo de corriente que pedirá el motor y prolongar el tiempo de uso del mismo. Para ellos se considerará el uso de diferentes basterías, ya sea las conocidas como baterías de ciclo profundo, Litio y óxido de cobalto(LiCoO2), o bien Litio y fosfato de hierro(LiFePO4). \\
En el caso de las de ciclo profundo, la diferencia fundamental entre estas baterías y las de ciclo corto como las de un automotor, radica en el uso que uno hace de ellas. En el caso de los automotores, se necesita  mucha  corriente de arranque y que una batería pueda entregar esa energía en corto tiempo. 
Entre 300 y 600 amper en unos 3 a 5 segundos. Luego la batería se recarga rápidamente y no hace falta siquiera que esté presente ya que el alternador provee de toda la energía necesaria para el funcionamiento del sistema eléctrico automotor. Partiendo de una batería 100 porciento cargada, el consumo de energía que hemos hecho no supera el 5 porciento. 
De aquí que se necesite una batería de ciclo corto. En sistemas de iluminación, las cargas aplicadas a las baterías guardan cierta relación con su capacidad y además suelen 
ser muy pequeñas al respecto de la  capacidad  de  la batería. Se toma energía por debajo de la capacidad de la misma.  Ejemplo, para una batería de 65  amper, se toma 3 amper a lo largo de 10 horas. Estas baterías son denominadas de ciclo profundo ya que admiten ser descargadas aproximadamente hasta un 90 porciento. \\
Para las baterías de Litio y óxido de cobalto(LiCoO2) y Litio y fosfato de hierro(LiFePO4), las de cobalto son las comunes en el mercado, con gran capacidad y eficiencia en comparación con las de plomo, sin embargo cuentan con menor estabilidad pero mayor densidad energética que otras baterías de ion de Litio con otro compuestos. En el caso de la LiFePO4, esta cuenta con una menor densidad energética, pero ofrece tiempos de vida mas largos, mayor rango de corriente exigida y son considerablemente mas seguras.\\
Por útlimo en componentes para simular los luces se utilizarían LEDs de diferentes colores junto con botoneras,

*****************hablar de un potenciometro y del stm 32, pensar en un puentes h para los motores por el consumo de corriente********
*****************poner alguna forma de obtener un feedback del motor, como un encoder óptico o un sensor de efecto hall*******


\section{Solución Propuesta}
Aplicando lo visto a lo largo del curso en los 4 proyectos, tomar el microcontrolador STM32F4 y crear el sistema de control, para un motor más grande como el de una motocicleta y además tomar otras salidas de control para manejar señales de control, como ignición, luces y herramientas de señalización como un velocímetro.
Esto es lograble tomando en cuenta que en el laboratorio 4 se aprendió a controlar un motor DC pequeño, quedará extrapolar este  conocimiento a un motor más robusto y grande.
El manejo de luces se puede hacer con cualquiera de los pines de salida vistos en el laboratorio 2 y la lógica que se debe añadir es poder obtener la velocidad del motor por medio del controlador para desplegarlo en pantalla. Más aún para el manejo de la ignición se utilizará un sistema parecido al sistema de sensor táctil utilizado en el laboratorio 2, esto para enviar una señal "contacto" al microcontrolador y activar el paso de corriente al motor.

\section{Procedimiento}
\begin{itemize}
\item Comunicar el microcontrolador con un motor DC de ser posible tamaño apto para una motocicleta eléctrica sino uno semejante, como los
usados en el arcoslab. Esto se logrará usando un código similar al laboratorio 4.
\item Adaptar este código para que tome en cuenta una entrada de un sensor táctil o algún tipo de ignición similar a la de los carros.
\item Mostrar en pantalla velocidad del motor. Esto por medio de un programa de comunicación similar al del laboratorio 2.
\item Añadir cualidades físicas de una motocicleta de ser posible. Entre esto calza la posibilidad de tener una carcaza que dará soporte
al motor,las baterias y un espacio para el microcontrolador y la ignición.
\end{itemize}
\section{Observaciones y recomendaciones}
Recomendación leer la documentación del STM32F4 a fondo para determinar si es capaz de manejar las tensiones y las corrientes 
requeridas por el motor a utilizar y así saber si es necesario utilizar circuitos de protección y/o algún otra configuración
electrónica adicional.

\bibliographystyle{alpha} 

%estos son los links de donde salió casi todo panchis
%http://www.cavadevices.com/archivos/FOLLETOS/BATERIAS%20CICLO%20PROFUNDO.pdf baterias de ciclo profundo
% http://www.technologyreview.com/news/406928/making-electric-vehicles-practical/ moto eléctrica

%mae estee aca hay algunas otras refs (fuck wikipedia)
%https://en.wikipedia.org/wiki/Brushless_DC_electric_motor#Brushless_vs._brushed_motors
%http://www.edn.com/design/sensors/4406682/Brushless-DC-Motors---Part-I--Construction-and-Operating-Principles
%https://en.wikipedia.org/wiki/Lithium_iron_phosphate_battery
%

\bibliography{refs}



\end{document}
